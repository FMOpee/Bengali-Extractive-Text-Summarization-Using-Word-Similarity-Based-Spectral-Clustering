The results presented at Table \ref{tab:result_comparison-1}, \ref{tab:other_language} and Figure \ref{fig:radarchart} highlights the effectiveness of proposed WSbSC model for extractive text summarization in Bengali, as well as its adaptability to other low-resource languages. This section analyses the comparative results, the strengths and limitations of the proposed method, and potential areas for further research.\\

As evidenced by the results shown in Table~\ref{tab:result_comparison-1} and Figure~\ref{fig:radarchart}, the WSbSC model consistently outperforms other graph-based extractive text summarization models, namely BenSumm \cite{das-2022-tfidf}, LexRank \cite{Erkan-lexRank-2004}, and SASbSC \cite{roychowdhury-etal-2022-spectral-base}, across four datasets. The proposed model shows better performance compared to other three methods on Rouge-1, Rouge-2, Rouge-LCS metrics. This performance improvement can largely be attributed to the novel approach of calculating sentence similarity. While calculating sentence similarity, taking the geometric mean of individual similarity between word pairs overcomes the lack of local word correspondence faced by the averaging vector method \cite{roychowdhury-etal-2022-spectral-base}. The Gaussian kernel-based word similarity provides a precise semantic relationships between sentences which further contribute in the performance improvement. Another reason for performance improvement is the usage of spectral clustering which is very effective in capturing irregular cluster shapes.\\

WSbSC includes a novel strategy for calculating sentence similarity despite the existence of other popular methods for comparing two sets of vectors. Our proposed strategy is more suited for similarity calculation in the context of language than other strategies such as Earth Movers Distance (EMD) \cite{Rubner-19998-emd}, Hausdorff Distance \cite{hausdorff-1914-hausdorff-distance}, Procrustes Analysis \cite{Gower-1975-procrustes-distance}. EMD \cite{Rubner-19998-emd} and Procrustes Analysis \cite{Gower-1975-procrustes-distance} are two very computationally expensive method who also involve scaling or rotating vectors; irrelevant for word vectors due to not holding any semantic meaning. Another method, Hausdorff distance \cite{hausdorff-1914-hausdorff-distance} calculates the highest possible distance between vectors from two set. Although similarly expensive as WSbSC, it is easily influenced by outlier words due to only considering the worst case scenario.\\

On the other hand, the proposed method focuses on local vector similarity between two sets which is more important for words. The Gaussian similarity function captures the proximity of points smoothly, providing a continuous value for similarity between two words in a normalized way. Gaussian similarity is also robust against small outliers due to being a soft similarity measure. Taking geometric mean also helps smooth over similarity values for outlier words.\\

One of the key strengths of this proposed method is the reduction of redundancy, a common challenge in extractive summarization methods, by grouping semantically similar sentences together. The use of spectral clustering for the grouping task improves the performance by not assuming a specific cluster shape. Another key strength of WSbSC is the improved sentence similarity calculation technique over word averaging method \cite{roychowdhury-etal-2022-spectral-base}, which dilutes the semantic meaning of a sentence. Another key strength is the scalability of the method across languages by requiring very few language-specific resources. This scalability is demonstrated in the experiments with Hindi, Marathi, and Turkish languages (Table \ref{tab:other_language}).\\

Despite its advantages, the WSbSC model does face some challenges. The model heavily relies on pre-trained word embeddings, which may not always capture the full details of certain domains or newly coined terms. The FastText \cite{grave-etal-2018-fasttext} dataset used here is heavily reliant on wikipedia for training which could introduce some small unforeseen biases. Where the word embeddings do not have some words of a given document, the model’s performance could degrade as it leaves those words out of the calculation process. The model also does not take into account the order in which words appear in a sentence or when they form special noun or verb groups. So it can be a little naive in some highly specialized fields.\\

The proposed WSbSC model represents a significant advancement in Bengali extractive text summarization due to its ability to accurately capture semantic similarity, reduce redundancy, maximize coverage and generalize across languages. While there are still challenges to be addressed, the results of this study demonstrate the robustness and adaptability of the WSbSC model, offering a promising direction for future research in multilingual extractive summarization.
