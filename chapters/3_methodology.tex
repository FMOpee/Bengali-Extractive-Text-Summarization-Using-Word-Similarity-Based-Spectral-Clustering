The summarization process followed here can be boiled down as two basic steps, grouping all the close sentences together based on their meaning to minimize redundancy and picking one sentence from each group to maximize sentence coverage. To group semantically similar sentences into clusters, we build a sentence similarity graph and perform spectral clustering on it~\cite{roychowdhury-etal-2022-spectral-base}. The sentence similarity graph is produced using a novel sentence similarity calculation algorithm that uses geometric mean of Gaussian similarity between individual word pairs from the two sentences. The Gaussian similarity is calculated using the vector embedding representations of the words. On the other hand, we used TF-IDF scores of the sentences in a cluster to pick the highest ranked and thus most representative sentence from the cluster~\cite{Akter-2017-tfidf-3,das-2022-tfidf,sarkar-2012-tfidf,sarkar-2012-tfidf-2}. The summarization process followed here involves 4 steps. These are, Pre-processing, Sentence similarity calculation, Clustering and Summary generation. These steps are further discussed in the following subsections. 

\subsection{Preprocessing}\label{subsec:preprocessing}
Preprocessing is the standard process of NLP that transforms raw human language inputs into a format that can be used by a computer algorithm. In this preprocessing step, the input document is transformed into a few set of vectors where each word is represented with a vector, each sentence is represented as a set of vectors and the whole document as a list containing those sets. To achieve this representation, the preprocessing follows three steps. These are tokenization, stop word removal, and word embedding. A very common step in preprocessing, word stemming, isn't used here as the word embedding dataset works best for the whole word instead of the root word. These steps are further discussed below.\\

Tokenization is the step of dividing an input document into sentences and words to transform it into a more usable format. Here, the input document is represented as a list of sentence and the sentences are represented as a list of words. Stop words, such as prepositions and conjunctions, add sentence fluidity but don’t carry significant meaning. Removing these words allows the algorithm to focus on more impactful words. To remove these stop words, we used a list\footnote{\textit{https://www.ranks.nl/stopwords/bengali}} of 363 bengali words. Word Embedding is the process of representing  words as vector in a vector space. In this vector space, semantically similar words are placed closer together so that the similarity relation between words can be expressed in an abstract mathematical way. We used the FastText dataset\footnote{\textit{https://fasttext.cc/docs/en/crawl-vectors.html}}~\cite{grave-etal-2018-fasttext} with 1.47 million Bengali words and their vector representation to achieve this step. Each word from the tokenized and filtered list is replaced with their corresponding vectors. If some words aren't present in the dataset, they are considered too rare and ignored. Following these steps, the input document is transformed into a set of vectors to be used in sentence similarity calculation. 

\subsection{Sentence Similarity Calculation}\label{subsec:sentence-similarity-calculation}
Sentence similarity is the key criteria to build a graphical representation of the relation between the sentences in the input document. This graphical representation is expressed via an affinity matrix to cluster the sentences into groups of semantically similar sentences. The nodes in the affinity matrix represents the sentences of the input and the edges of the matrix represents the similarity between two sentence to graphically represent the input document. Here, we proposed a novel sentence similarity calculation technique using individual Gaussian similarity between close word pairs to construct the affinity matrix. To calculate the sentence similarity between two sentences, we adhere to the following steps.\\

Firstly, for every word of a sentence, we find its closest counterpart from the other sentence to build a word pair. The Euclidean distance between the vector representation of the two words from this pair is defined as the Most Similar Word Distance ($D_{msw}$) to be used in the following steps. The process of calculating the $D_{msw}$ is shown in the equation \ref{eq:msd}. In this equation, for every word vector $x$, in a sentence $X$, we find the Euclidean distance ( $d(x,y_i)$ ) between the word vectors $x$ and $y_i$ where $y_i$ is a word vector from the sentence $Y$. The lowest possible distance in this set of Euclidean distance is the $D_{msw}$. 
\begin{equation}\label{eq:msd}
    D_{msw}(x,Y) = min(\{d(x,y_i) : y_i \in Y \})
\end{equation}
Then, we calculate the $D_{msw}$ for every word of the two sentences $X$ and $Y$ to make the sentence similarity calculation symmetric. This process is shown in the equation \ref{eq:mswdset}. In this equation, $D$, is a set containing all the $D_{msw}$ from both $X$ and $Y$ that would be used in the later steps.
\begin{equation}
    D = \{D_{msw}(x,Y) : x \in X\} \cup \{D_{msw}(y,X) : y \in Y\}
    \label{eq:mswdset}
\end{equation}
After this, the word similarity between the word pairs is calculated to get the degree of word correspondence between the two sentences. Here, the word similarity is calculated using Gaussian kernel function for the elements of the set $D$ because Gaussian kernel functions provides a smooth, flexible and most representative similarity between two vectors \cite{babud-1986-gaussian}. The process of calculating word similarity ($W_{sim}$) is given in the following equation \ref{eq:wsim}. In this equation, for every element $D_i$ in set $D$, we calculate the Gaussian similarity to obtain word similarity. In the formula for Gaussian similarity, $\sigma$ denotes the standard deviation that can be used as a control variable. The standard deviation represents the blurring effect of the kernel function. A lower value for $\sigma$ represents a high noise sensitivity of the function \cite{babud-1986-gaussian}. The value of sigma was fine-tuned to be $5\times10^{-11}$ which gives the similarity measurement. The process of fine-tuning is described in the experimentation section (section \ref{subsubsec:sigma}). 
\begin{equation}\label{eq:wsim}
    W_{sim} = \{ exp\left(\frac{-D_i^2}{2\sigma^2}\right) : D_i \in D\}
\end{equation}
Finally, the sentence similarity between the two sentences $Sim(X,Y)$ is calculated using geometric mean of the word similarities from the above step to construct an affinity matrix. Using geometric mean makes the similarity value less prone to effects of outliers to make the calculation more reliable. This process is shown in the following equation \ref{eq:sent_sim}. In this equation, the geometric mean of each $w_{sim}$ value for the two sentences is simplified in the equation \ref{eq:sent_sim} to make the calculation process more computation friendly. 
\begin{equation}\label{eq:sent_sim}
    \begin{split}
        Sim(X,Y)
        &=  \left(
                \prod_{i=1}^nW_{Sim_i}
            \right)^{\frac{1}{n}}\\
        &=  \left(
                e^{\frac{-D_1^2}{2\sigma^2}}\cdot
                e^{\frac{-D_2^2}{2\sigma^2}}\cdot
                    \ldots \cdot
                e^{\frac{-D_n^2}{2\sigma^2}}
            \right)^\frac{1}{n}\\
        &=  exp\left(
                -\frac{D_1^2+D_2^2+\ldots+D_n^2}{2n\sigma^2}
            \right)\\
        &=  exp\left(
                -\frac{\sum_{i=1}^nD_i^2}{2n\sigma^2}
            \right)
    \end{split}
\end{equation}
By following steps described above, we get a similarity value for two sentence. This value solves the lack of local word correspondence problem faced by the word averaging method based similarity calculation \cite{roychowdhury-etal-2022-spectral-base} by considering local word to word similarity. This correspondence is shown in the figure \ref{fig:msd} to demonstrate the merit of the method described above. Figure \ref{fig:msd} follows up the scenario from the figure \ref{fig:sarkar-problem} to show that the proposed method can solve the local word correspondence problem faced by word averaging method. The figure shows that the scenario \ref{fig:msd}(a) has a set of smaller $D_{msw}$ than the scenario \ref{fig:msd}(b). Having smaller $D_{msw}$ makes the individual word similarities $W_{sim}$ from equation \ref{eq:wsim} larger due to the nature of Gaussian kernel function. These values would result in a higher sentence similarity for the sentences in the scenario \ref{fig:msd}(a) than in the scenario \ref{fig:msd}(b). This solved the problem showed in the figure \ref{fig:sarkar-problem} where the scenario \ref{fig:sarkar-problem}(a) has a larger sentence average distance than \ref{fig:sarkar-problem}(b) resulting in \ref{fig:sarkar-problem}(a) having a smaller sentence similarity than \ref{fig:sarkar-problem}(b).\\

The whole process of sentence similarity calculation is shown in the following algorithm \ref{alg:similarity}. In this algorithm, we calculate an affinity matrix using the input word vector list from the preprocessing step. We took the most similar word distance $D_{msw}$ in line 8--13 and 18--23 for each word (line 7 and 17) of a sentence pair (line 3 and 6). $\sum^n_{i=1}D_i^2$ from equation \ref{eq:sent_sim} is calculated in the lines 14 and 24 to be used in the calculation of sentence similarity (line 27). The similarity is used to construct an affinity matrix $A$ (line 28).
\begin{figure}
    \centering
    \begin{tikzpicture}[scale=0.00125\textwidth]

    \def\xleft{2};
    \def\xright{14};
    % Draw quarters
    \draw (\xleft,0) rectangle (\xleft+10,10);
    \draw (\xright,0) rectangle (\xright+10,10);

    % Add labels
    \node at (0.95,5.15) {(a)};
    \node at (12.95,5.15) {(b)};

    %set a
    \def\aax{0.25};\def\aay{0.35};
    \def\abx{0.25};\def\aby{5.8};
    \def\acx{5};\def\acy{0.5};
    \def\adx{4.5};\def\ady{9.5};
    \def\aex{5.5};\def\aey{9.6};
    \def\afx{9.5};\def\afy{0.5};
    %set b
    \def\bax{1.5};\def\bay{4};
    \def\bbx{4.5};\def\bby{8.5};
    \def\bcx{6.5};\def\bcy{1};
    \def\bdx{8.2};\def\bdy{7.8};
    %set c
    \def\cax{2.8};\def\cay{4};
    \def\cbx{4.2};\def\cby{6.2};
    \def\ccx{6.2};\def\ccy{5};
    \def\cdx{6.5};\def\cdy{4};



    %set a at a
    \fill[blue] (\xleft+\aax,\aay) circle(.15);
    \fill[blue] (\xleft+\abx,\aby) circle(.15);
    \fill[blue] (\xleft+\acx,\acy) circle(.15);
    \fill[blue] (\xleft+\adx,\ady) circle(.15);
    \fill[blue] (\xleft+\aex,\aey) circle(.15);
    \fill[blue] (\xleft+\afx,\afy) circle(.15);
    %set a at b
    \fill[blue] (\xright+\aax,\aay) circle(.15);
    \fill[blue] (\xright+\abx,\aby) circle(.15);
    \fill[blue] (\xright+\acx,\acy) circle(.15);
    \fill[blue] (\xright+\adx,\ady) circle(.15);
    \fill[blue] (\xright+\aex,\aey) circle(.15);
    \fill[blue] (\xright+\afx,\afy) circle(.15);

     %set b at a
    \fill[red] (\xleft+\bax,\bay) circle(.15);
    \fill[red] (\xleft+\bbx,\bby) circle(.15);
    \fill[red] (\xleft+\bcx,\bcy) circle(.15);
    \fill[red] (\xleft+\bdx,\bdy) circle(.15);

     %set c at b
    \fill[red] (\xright+\cax,\cay) circle(.15);
    \fill[red] (\xright+\cbx,\cby) circle(.15);
    \fill[red] (\xright+\ccx,\ccy) circle(.15);
    \fill[red] (\xright+\cdx,\cdy) circle(.15);


    %drawing line a
    \draw[-{Latex[length=2mm]},dashed] (\xleft+\aax,\aay) -- (\xleft+\bax,\bay);
    \draw[{Latex[length=2mm]}-{Latex[length=2mm]},dashed] (\xleft+\bax,\bay) -- (\xleft+\abx,\aby);
    \draw[{Latex[length=2mm]}-{Latex[length=2mm]},dashed] (\xleft+\bbx,\bby) -- (\xleft+\adx,\ady);
    \draw[{Latex[length=2mm]}-{Latex[length=2mm]},dashed] (\xleft+\bcx,\bcy) -- (\xleft+\acx,\acy);
    \draw[-{Latex[length=2mm]},dashed] (\xleft+\aex,\aey) -- (\xleft+\bbx,\bby);
    \draw[-{Latex[length=2mm]},dashed] (\xleft+\bdx,\bdy) -- (\xleft+\aex,\aey);
    \draw[-{Latex[length=2mm]},dashed] (\xleft+\afx,\afy) -- (\xleft+\bcx,\bcy);

    %drawing line b
    \draw[{Latex[length=2mm]}-{Latex[length=2mm]},dashed] (\xright+\abx,\aby) -- (\xright+\cax,\cay);
    \draw[{Latex[length=2mm]}-{Latex[length=2mm]},dashed] (\xright+\adx,\ady) -- (\xright+\cbx,\cby);
    \draw[{Latex[length=2mm]}-{Latex[length=2mm]},dashed] (\xright+\cdx,\cdy) -- (\xright+\acx,\acy);
    \draw[-{Latex[length=2mm]},dashed] (\xright+\aax,\aay) -- (\xright+\cax,\cay);
    \draw[-{Latex[length=2mm]},dashed] (\xright+\aex,\aey) -- (\xright+\cbx,\cby);
    \draw[-{Latex[length=2mm]},dashed] (\xright+\ccx,\ccy) -- (\xright+\aex,\aey);
    \draw[-{Latex[length=2mm]},dashed] (\xright+\afx,\afy) -- (\xright+\cdx,\cdy);

\end{tikzpicture}
    \caption{Local word correspondence of $D_{msw}$ method. Dots in the figure represent word vectors and are coloured with the same colour if they are from the same sentence. Black dashed arrows represent the $D_{msw}$ from its destination word vector. Here, scenario (a) will have larger similarity due to it having smaller $D_{msw}$ than scenario (b)}
    \label{fig:msd}
\end{figure}
\begin{algorithm} \caption{Sentence Similarity Calculation} \label{alg:similarity}
\begin{algorithmic}[1]
    \State $l \gets$ length(WordVectorList)
    \State $A \gets [ [ 0 ] * l ] * l$
    \For{each sentence$_i$ in WordVectorList}
        \State $D_{Square} \gets 0$
        \State n $\gets 0$
        \For{each sentence$_j$ in WordVectorList}
            \For{each word$_i$ in sentence$_i$}
                \State $D_{msw} \gets \infty$
                \For{each word$_j$ in sentence$_j$}
                    \If{Distance(word$_i$, word$_j$) $< D_{msw}$}
                        \State $D_{msw} \gets$ distance(word$_i$, word$_j$)
                    \EndIf
                \EndFor
                \State $D_{Square} \gets D_{Square} + D_{msw}^2$
                \State $n \gets n+1$ 
            \EndFor
            \For{each word$_j$ in sentence$_j$}
                \State $D_{msw} \gets \infty$
                \For{each word$_i$ in sentence$_i$}
                    \If{Distance(word$_i$, word$_j$) $< D_{msw}$}
                        \State $D_{msw} \gets$ distance(word$_i$, word$_j$)
                    \EndIf
                \EndFor
                \State $D_{Square} \gets D_{Square} + D_{msw}^2$
                \State $n \gets n+1$
            \EndFor
            \State similarity $\gets \exp \left( \frac{- D_{Square}}{2 \times n \times \sigma^2} \right)$
            \State $A[i][j] \gets A[j][i] \gets$ similarity
        \EndFor
    \EndFor
    \State \textbf{Return} $A$
\end{algorithmic}
\end{algorithm}

\subsection{Clustering}\label{subsec:clustering}
Clustering is a key corner stone of the proposed method where we aim to cluster semantically similar sentences together to divide the input document into multiple topics. Clustering the document helps with minimizing redundancy in the output summary by not selecting multiple sentences from the same topic. For clustering, spectral and DBSCAN clustering methods were considered due to their capability of being able to cluster irregular shapes. In clustering sentences, spectral clustering was found to perform better than DBSCAN because smaller input documents have lower density which hinders DBSCAN \cite{roychowdhury-etal-2022-spectral-base}.\\

Spectral clustering takes the affinity matrix of a graph as input and returns the grouping of graph nodes by transforming the graph into its eigenspace \cite{vonLuxburg-2007-spectral-tutorial}. The following equation \ref{eq:affinity} shows the process of building an affinity matrix. In this equation, for every sentence pair $S_i$ and $S_j$, we calculate their sentence similarity using the equation \ref{eq:sent_sim} and use the value in both $A_{ij}$ and $A_{ji}$ place of the affinity matrix $A$.
\begin{equation}\label{eq:affinity}
    A_{ij}=A_{ji}=Sim(S_i,S_j)
\end{equation}
The affinity matrix is clustered into a reasonable, $k=\lceil\frac{N}{5}\rceil$ groups to achieve an output summary which short while the sentence groups resulting from clustering is also not too broad.

\subsection{Summary Generation}\label{subsec:summary-generation}
Output summary is generated by selecting one sentence from each cluster achieved in the previous step to minimize topic redundancy and to maximize topic coverage. To select one sentence from a cluster, we perform TF-IDF ranking on the sentences inside a cluster and pick the sentence with the highest TF-IDF score. To get the TF-IDF score of a sentence, we take the sum of all TF-IDF values for the words in that sentence. The TF-IDF value for a word is achieved by multiplying how many time the word appeared in the input document (Term Frequency, TF) and the inverse of how many document does the word appear in a corpus (Inverse Document Frequency, IDF). The process of scoring sentences are shown in the following equation \ref{eq:tfidf}. In this equation, for each word $W_i$ in a sentence $S$ and a corpus $C$, we calculate the TF-IDF score of a sentence.
\begin{equation}\label{eq:tfidf}
	\text{TFIDF}(S) = \sum_{i=1}^{\text{length}(S)}\text{TF}(W_i) \times \text{IDF}(W_i,C)	
\end{equation}
The sentences with the best TF-IDF score from each clusters are then compiled as the output summary in their order of appearance in the input document to preserve the original flow of information. The process of generating output summary is further expanded in the following algorithm \ref{alg:summary}. After the clustering step (line 2), we took the TF-IDF score (line 7) of each sentence in a cluster (line 6). For each cluster (line 4), we pick the best scoring sentence (line 9). These sentences are then ordered (line 11) and concatenated (line 13--15) to generate the output summary.

\begin{algorithm} \caption{Summary Generation} \label{alg:summary}
\begin{algorithmic}[1]
    \State $k \gets \lceil$ length($A$) / 5 $\rceil$
    \State clusters $\gets$ spectral\_clustering(adjacency = $A$, $k$)
    \State indexes $\gets \{\}$
    \For{each cluster$_i$ in clusters}
        \State TFIDF $\gets \{\}$
        \For{each index in cluster$_i$}
            \State TFIDF.append(tfidf\_sum(sentences[index]))
        \EndFor
        \State indexes.append(indexof(max(TFIDF)))
    \EndFor
    \State sort(indexes)
    \State $S \gets `` "$
    \For{each $i$ in indexes}
        \State $S \gets S +$ sentences[$i$]
    \EndFor
    \State \textbf{Return} $S$
\end{algorithmic}
\end{algorithm}

%\subsection{Implementation Details}\label{subsec:implementation}
%The implementation details of the proposed method is provided here for the purpose of reproducibility. While implementing the preprocessing step, we used the nltk package \cite{Bird-2009-nltk} to tokenize the input document. To remove stop words from the tokenized document, we used regex matching in a dataset of 363 Bengali stop words.  

