The summarization process followed here can be boiled down as, grouping together all the close sentences
together based on their meaning and then picking one sentence from each group to minimize redundancy
and maximize sentence coverage.
Often most widely used extractive summarization methods involve scoring the sentences based on some metric and
then picking the best scoring sentences to generate the summaries~\cite{das-2022-tfidf,sarkar-2012-tfidf}.
But in a single topic text document it often picks similar sentences creating redundancy due to all of them having the
central topic terms.
To mitigate this, firstly grouped the sentences with similar meaning together and then finally picked one
sentence from each group to generate a summary.
This will ensure maximum coverage of topics while also reducing redundancy.
This method has also been tried before by \citeauthor{roychowdhury-etal-2022-spectral-base}~\cite{roychowdhury-etal-2022-spectral-base}.
But the main challenge to reducing the redundancy is to develop a method that can accurately predict how close the
meaning of the two sentences are.
In this paper, we propose a method that can do it.
The summarization process followed here involves total 4 steps.
These are, in order of their use, Pre-processing, Sentence similarity calculation, Clustering and Summary generation.
These steps are further discussed in the following subsections.

\subsection{Pre-processing}\label{subsec:pre-processing}
Pre-processing is a standard step in Natural Language Processing that transforms the raw human language
inputs into a format that can be used by a computer algorithm.
Here the input document is transformed into a list of sets of vectors where each word is represented
with a vector, each sentence as a set of vectors and the whole document as a list of said sets.
The Pre-processing involves 3 steps.
These are Tokenization, Stop Word Removal, Word Embedding.
A very common step in Pre-processing, Word Stemming, isn't used in here as the word embedding dataset
works best for the whole word instead of the stemmed word.
These steps are further discussed bellow.
This whole process is shown in Algorithm~\ref{alg:preprocessing}.

\subsubsection{Tokenization}
Tokenization is the step of dividing an input document into sentences and words in a usable format.
Here the input document was firstly divided into sentences by using the NLTK library~\cite{Bird-2009-nltk} of
python by using regex matching.
These sentences are then further divided into words and are put together in a list.
Then these lists are further compiled into a list of list for the whole input document.
An example of performing this step is given bellow.\\

\textbf{Before Tokenization:}\\
{\bng
    ra\*sh*iya-IU\*kR*en Jud/dh shuru HOyar por ma\*r/k*in Dola\*r*er \*b*i\*n*imoy Har \*b*e\*rh*e\*ch*e.
    % রাশিয়া    -ইউক্রেন    যুদ্ধ     শুরু   হওয়ার  পর  মার্কিন       ডলারের      বিনিময়        হার  বেড়েছে।
    E\*t*e Amda\*n*i\*n*ir/bhor \*d*eshgul \*b*ipa\*k*e po\*rh*e\*ch*e.
    % এতে  আমদানিনির্ভর            দেশগুল          বিপাকে        পড়েছে।
    \*b*{oi}\*d*e\*sh*ik mudRar \*r*ijar/bh ba mojut ko\*m*e JaOyay baNNGla\*d*esh\*k*eO Amda\*n*i sii\*m*it
    % বৈদেশিক             মুদ্রার    রিজার্ভ       বা  মজুত  কমে    যাওয়ায়    বাংলাদেশকেও           আমদানি    সীমিত
    kor\*t*e Ho\*y*e\*ch*e.
    % করতে     হয়েছে
}\\

\textbf{After Tokenization:}\\
{\bng
    ((ra\*sh*iya, IU\*kR*en, Jud/dh, shuru, HOyar, por, ma\*r/k*in, Dola\*r*er, \*b*i\*n*imoy, Har, \*b*e\*rh*e\*ch*e),
    (E\*t*e, Amda\*n*i\*n*ir/bhor, \*d*eshgul, \*b*ipa\*k*e, po\*rh*e\*ch*e),
    (\*b*{oi}\*d*e\*sh*ik, mudRar, \*r*ijar/bh, ba, mojut, ko\*m*e, JaOyay, baNNGla\*d*esh\*k*eO, Amda\*n*i,
    sii\*m*it, kor\*t*e, Ho\*y*e\*ch*e))
}

\subsubsection{Stop Word Removal}

Stop words, such as prepositions and conjunctions, add sentence fluidity but don’t carry significant meaning.
For Bengali, a commonly used dataset\footnote{\textit{https://www.ranks.nl/stopwords/bengali}} of 363 Bengali stop
words used to remove the stop words on a matching basis.
After completing this step the input document would look like this:\\
{\bng
    ((ra\*sh*iya, IU\*kR*en, Jud/dh, shuru, ma\*r/k*in, Dola\*r*er, \*b*i\*n*imoy, Har, \*b*e\*rh*e\*ch*e),
    (Amda\*n*i\*n*ir/bhor, \*d*eshgul, \*b*ipa\*k*e, po\*rh*e\*ch*e),
    (\*b*{oi}\*d*e\*sh*ik, mudRar, \*r*ijar/bh, mojut, ko\*m*e, JaOyay, baNNGla\*d*esh\*k*eO, Amda\*n*i, sii\*m*it))
}\\
Here the removed stop words are {\bng HOyar, por, E\*t*e, ba, kor\*t*e, Ho\*y*e\*ch*e}

\subsubsection{Word Embedding}
Word Embedding is a step that replaces the words in a sentence with a corresponding vector in a vector
space such that the closer two words are in terms of meaning to one another the smaller the distance
of those vectors would be.
%It means that, the Euclidean distance between the points would be smaller.
To achieve this step, we used a dataset with 1.47 million Bengali words produced by \citeauthor{grave-etal-2018-fasttext}
\cite{grave-etal-2018-fasttext} crawling Wikipedia and other online resources to make the word embedding vectors.
Finally, each word that is present in the tokenized and filtered list is replaced with their
corresponding vectors and the words that isn't found is ignored and considered to be too rare to be relevant.

\begin{algorithm} \caption{Preprocessing} \label{alg:preprocessing}
\begin{algorithmic}[1]
    \State SentenceList $\gets$ tokenize($D$)
    \For{each sentence $i$ in SentenceList}
        \State WordList.append(tokenize(sentence$_i$))
    \EndFor
    \State VectorList $\gets \{\}$
    \For{each WordList$_i$ in WordList}
        \State VectorList$_i \gets \{\}$
        \For{each Word in WordList$_i$}
            \If{Word $\in$ StopWordList}
                \State WordList$_i$.remove(Word)
            \ElsIf{Word $\in$ VectorEmbedding.keys()}
                \State VectorList$_i$.append(VectorEmbedding.get(Word))
            \Else
                \State WordList$_i$.remove(Word)
            \EndIf
        \EndFor
        \State VectorList.append(VectorList$_i$)
    \EndFor
    \State \textbf{Return} VectorList
\end{algorithmic}
\end{algorithm}

\subsection{Sentence Similarity Calculation}\label{subsec:sentence-similarity-calculation}
To perform clustering in a graph, an affinity matrix is needed.
A similarity calculation technique using individual word distance and
Gaussian similarity have been proposed here.
The process is shown in Algorithm~\ref{alg:similarity}.
A previous similar method by \citeauthor{roychowdhury-etal-2022-spectral-base}~\cite{roychowdhury-etal-2022-spectral-base}
had averaged all the vectors present in a sentence to get a vector for the sentence.
But this is not rally a sound strategy as words in a sentence are generally complementary instead of being similar.
So the word vectors tend to scatter around in the vector space rather than being grouped closed together.
This leads to the average word vector having a tendency towards the center of the vector space.
This would lead to the affinity between these vectors not being representative of the actual meaning.\\

To mitigate this, in this study, similarity between individual words in a pair of sentences have been considered.
For this, firstly, the Most Similar Word Distance ($D_{msw}$) have to be calculated as shown in Equation~\ref{eq:msd}.
The $D_{msw}$ denotes the distance between a word vector and the word vector that
is closest to its meaning from the other sentence.

\begin{equation}\label{eq:msd}
    D_{msw}(x,Y) = min(\{d(x,y_i) : y_i \in Y \})
\end{equation}

All the $D_{msw}$ for each word in each sentence are then put together in a list like shown in
Equation~\ref{eq:mswdset}.

\begin{equation}
    D_{msw} = \{D_{msw}(x,Y) : x \in X\} \cup \{D_{msw}(y,X) : y \in Y\}
    \label{eq:mswdset}
\end{equation}

Here, for every word vector $x$ in a sentence $X$, the closest vector, in terms of
Euclidean distance denoted by $d(x,y)$, is identified from all the word vectors $y$ in the sentence $Y$.
Secondly, the word similarity is calculated using Gaussian similarity for each of these $D_{msw}$.
The equation involved is shown in Equation~\ref{eq:wsim}.

\begin{equation}\label{eq:wsim}
    WSim_i = e^{\frac{-D_{msw_i}^2}{2\sigma^2}}
\end{equation}

The Sentence similarity between the two sentence or $Sim(X,Y)$ is calculated as the Geometric
mean of all the word similarities from both sentence so that the similarity between two sentence is symmetric.
This is explained in the Equation~\ref{eq:sent_sim}

\begin{equation}\label{eq:sent_sim}
    \begin{split}
        Sim(X,Y)
        &=  \left(
                \prod_{i=1}^nWSim_i
            \right)^{\frac{1}{n}}\\
        &=  \left(
                e^{\frac{-D_{msw_1}^2}{2\sigma^2}}\cdot
                e^{\frac{-D_{msw_2}^2}{2\sigma^2}}\cdot
                    \ldots \cdot
                e^{\frac{-D_{msw_n}^2}{2\sigma^2}}
            \right)^\frac{1}{n}\\
        &=  exp\left(
                -\frac{D_{msw_1}^2+D_{msw_2}^2+\ldots+D_{msw_n}^2}{2n\sigma^2}
            \right)\\
        &=  exp\left(
                -\frac{\sum_{i=1}^nD_{msw_i}^2}{2n\sigma^2}
            \right)
    \end{split}
\end{equation}

Here, by taking geometric mean of the similarity between the closest words together in two sentence,
an effective word to word comparison has been created between those sentences.
This reduces any misleading distance that would have come from the word averaging method due to
the tendency towards center.
An example of this can be seen at Figure~\ref{fig:msd}.
Here, a more representative word association can be seen for both scenarios from Figure~\ref{fig:sarkar-problem}.
Red and Blue dots in the figure represent two set of word vectors in a sentence pair.
Black dashed lines show the Most Similar Distance $(D_{msw}(x,Y))$ for a word vector $x$ and the other sentence $Y$.
The arrowheads point from $x$.
The Figure~\ref{fig:msd}(a) shows the $D_{msw}$ for Scenario A in Figure~\ref{fig:sarkar-problem}(a).
The Figure~\ref{fig:msd}(b) Shows the $D_{msw}$ for Scenario B in Figure~\ref{fig:sarkar-problem}(b).
We can see that problem caused by the averaging method have been mitigated here.\\

\begin{figure}
    \centering
    \input{figs/msd}
    \caption{Process of obtaining $D_{msw}$}
    \label{fig:msd}
\end{figure}

The similarity equation (equation-\ref{eq:sent_sim}) has a standard deviation $\sigma$ which works
as a control variable and was fine-tuned to be $5 \times 10^{-11}$ where it gave the best results.

\begin{algorithm} \caption{Sentence Similarity Calculation} \label{alg:similarity}
\begin{algorithmic}[1]
    \State $n \gets$ length($VL$)
    \State $A \gets \{ \{0\} \times n \} \times n$
    \For{each sentence$_i$ in $VL$}
        \State $D_{\text{Square}} \gets 0$
        \State count $\gets 0$
        \For{each sentence$_j$ in $VL$}
            \For{each word$_i$ in sentence$_i$}
                \State $D_{\text{msw}} \gets \infty$
                \For{each word$_j$ in sentence$_j$}
                    \If{Distance(word$_i$, word$_j$) $< D_{\text{msw}}$}
                        \State $D_{\text{msw}} \gets$ Distance(word$_i$, word$_j$)
                    \EndIf
                \EndFor
                \State $D_{\text{Square}} \gets D_{\text{Square}} + D_{\text{msw}}^2$
                \State count++
            \EndFor
            \For{each word$_j$ in sentence$_j$}
                \State $D_{\text{msw}} \gets \infty$
                \For{each word$_i$ in sentence$_i$}
                    \If{Distance(word$_i$, word$_j$) $< D_{\text{msw}}$}
                        \State $D_{\text{msw}} \gets$ Distance(word$_i$, word$_j$)
                    \EndIf
                \EndFor
                \State $D_{\text{Square}} \gets D_{\text{Square}} + D_{\text{msw}}^2$
                \State count++
            \EndFor
            \State similarity $\gets \exp \left( \frac{- D_{\text{Square}}}{2 \times \text{count} \times \sigma^2} \right)$
            \State $A[i][j] \gets A[j][i] \gets$ similarity
        \EndFor
    \EndFor
    \State \textbf{Return} $A$
\end{algorithmic}
\end{algorithm}
%\subsubsection{Viability of this strategy}
%To get the similarity between two sentence, we needed a strategy that can match two sets of vectors.
%Other popular methods for the same functionality are also available.
%For example, Earth Mover's Distance (EMD)~\cite{Rubner-19998-emd}, Hausdorff distance, Procrustes Analysis etc.
%EMD tries to find the lowest amount of ``work'' needed to transform one set into the other one.
%This is very computationally expensive.
%And also it focuses on scaling and rotating which are not relevant in word vector space.
%Hausdorff distance takes the worst case scenario and is easily influenced by outliers.
%So, it was avoided.
%Procrustes Analysis tries to find the lowest amount of misalignment after scaling and rotating.
%Both of them are irrelevant in word vector space.\\
%
%On the other hand, our method focuses on Local Point Correspondence between two sets which is more
%important for words.
%The Gaussian similarity function captures the proximity of points smoothly,
%providing continuous feedback on how similar two points are in a normalized way.
%It is also robust against small perturbations because of the use of a soft similarity measure (Gaussian)
%and geometric mean which helps smooth over small differences in point locations.

\subsection{Clustering}\label{subsec:clustering}
The clustering is the most integral part of this summarization technique, aiming to group all the
sentences with similar meanings together.
Here, spectral clustering is used to cluster the sentences using sentence similarity calculated in the step above.
Spectral clustering was chosen here because \citeauthor{roychowdhury-etal-2022-spectral-base}
\cite{roychowdhury-etal-2022-spectral-base} found it to be better performing than DBSCAN method.
The spectral clustering steps were followed according to the tutorial given by
\cite{vonLuxburg-2007-spectral-tutorial}. \\

To perform spectral clustering on a data, firstly, an affinity matrix is required that shows
the weight of edges between the vertexes in the graph.
Here the affinity $A$ is prepared using the following Equation~\ref{eq:affinity}.

\begin{equation}\label{eq:affinity}
    A_{ij}=A_{ji}=Sim(S_i,S_j)
\end{equation}

Here, $S_i, S_j$ are sentences from the input document.
The affinity matrix, $A$, is used in the spectral clustering implementation in the SciKit-learn
tool~\cite{Pedregosa-2011-scikit-learn} in python.
Here, to use the function, we also need to provide the number of clusters to achieve.
The number of clusters are fixed at $k=ceiling\left(\frac{N}{5}\right)$ due to it being a
reasonable size to contain all necessary sentences as well as being short enough to be an effective summary.

\subsection{Summary Generation}\label{subsec:summary-generation}
After clustering, we pick one sentence from each cluster.
The sentences inside a cluster are ranked among themselves using TF-IDF .
The best ranked sentence of the clusters by their order of TF-IDF score is
selected from each cluster.
We then rearranged these picked sentences are in their order of appearance to retain the normal flow of
information in the input.
These sentences are then concatenated together to produce the final output summary.
This is shown in Algorithm~\ref{alg:summary}

\begin{algorithm} \caption{Summary Generation} \label{alg:summary}
\begin{algorithmic}[1]
    \State $k \gets \lceil$ length($A$) / 5 $\rceil$
    \State clusters $\gets$ spectral\_clustering(adjacency = $A$, $k$)
    \State indexes $\gets \{\}$
    \For{each cluster$_i$ in clusters}
        \State TFIDF $\gets \{\}$
        \For{each index in cluster$_i$}
            \State TFIDF.append(tfidf\_sum(sentences[index]))
        \EndFor
        \State indexes.append(indexof(max(TFIDF)))
    \EndFor
    \State sort(indexes)
    \State $S \gets ""$
    \For{each $i$ in indexes}
        \State $S \gets S +$ sentences[$i$]
    \EndFor
    \State \textbf{Return} $S$
\end{algorithmic}
\end{algorithm}
