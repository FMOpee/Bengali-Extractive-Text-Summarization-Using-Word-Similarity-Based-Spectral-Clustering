
In this study, we proposed and evaluated a Word Similarity-based Spectral Clustering (WSbSC)
method for Bengali extractive text summarization.
The method uses semantic relationships between words to
identify the best sentences from a text, addressing the need for
effective summarization techniques in the Bengali language,
which remains underrepresented in natural language processing research.
By using spectral clustering,
we aimed to group sentences based on their semantic similarity,
improving the coherence and relevance of the generated summaries.\\

Through extensive experimentation on different Bengali summarization datasets,
our results showed that the WSbSC method outperforms several baseline techniques,
particularly in grouping the sentences into key topics of documents.
Despite these promising results,
there are areas of further improvement.
One limitation observed is that the method may struggle with highly
specialized or domain-specific texts,
where deeper linguistic features beyond word similarity could be considered.
Future work could explore hybrid models that integrate other
post-processing techniques to improve the output.\\

In conclusion, this work contributes to the growing body of computational
linguistics research focused on low-resource languages like Bengali.
The WSbSC method offers a novel approach for extractive summarization
and sets the stage for further advancements in both Bengali text processing
and multilingual summarization techniques.


