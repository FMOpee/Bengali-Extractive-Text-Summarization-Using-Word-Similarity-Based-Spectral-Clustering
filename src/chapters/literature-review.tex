Automatic Text Summarization have been a problem people are trying to solve for a long time.
Attempts at automatic text summarization started with indexing based methods~\cite{Baxendale_1958_firstsummarization}.
In this attempt \citeauthor{Baxendale_1958_firstsummarization}~\cite{Baxendale_1958_firstsummarization} attempted to
summarize by using a sentence scoring system where if sentences had some certain words, they would be scored higher.
But methods like this failed to capture the topic and essence of the input text.
To solve this, Text Summarization with statistical methods like TF-IDF became very popular.
\citeauthor{edmundson_1969_earlysum}~\cite{edmundson_1969_earlysum} proposed their method which is also called
Topic-based approach because it can focus on the central topic of a document.
It uses two metric, Term Frequency (how many times a term appear in the input) and Inverse Document Frequency
(inverse of how many documents the term appears in a corpus) to calculate the importance of a term in a document.
This method identifies the words that are common in the input text but not as common in the language and identifying
them as the central topic.
But it was too prone to error due to it thinking every word as a unique isolated term and not having
any semantic relation with other words.
Some words may be a central topic of a document but not identified as such because they got divided into too many
synonyms. \\

Modern breakthroughs into the extractive text summarization began with the usage of Graph-based Extractive
Text Summarization methods~\cite{Erkan-lexRank-2004, mihalcea-2004-textrank}.
LexRank~\cite{Erkan-lexRank-2004} calculates the similarity between two sentences using cosine similarity and builds a graph
of all the similarity of every pair of sentence in the input.
The most important sentences are then identified using the PageRank~\cite{page-PageRank-1999} algorithm on the graph.
This algorithm ranked the sentences, who are most similar with other high ranked sentences, higher.
TextRank~\cite{mihalcea-2004-textrank} also uses a similar approach.
But for every sentence, the method distributed its scores to its neighbours using random walk.
The process was done over and over until the scores converge.
Although these models are very novel compared to their time, they still lacked fundamental understanding of the
words involved in a sentence.\\

Word Vector Embedding have been a novel concept for document abstraction for some time now.
The seminal work of \citeauthor{salton-1975-word-vector}~\cite{salton-1975-word-vector} have been pivotal in
conceptualising Word Vector Space where the closer two word are in the vector space, the closer they are semantically.
Using word vector for summarization have only been attempted recently~\cite{Jain-2017-word-vector-embedding-summary}.\\

Text Summarization attempts in Bengali is a more recent development than in other high resource languages.
Earlier Extractive methods have been focused on some derivative of TF-IDF based text
summarization~\cite{chowdhury-etal-2021-tfidf-clustering,das-2022-tfidf,sarkar-2012-tfidf}.
\citeauthor{sarkar-2012-tfidf}~\cite{sarkar-2012-tfidf} used simple TF-IDF score of each sentence to rank them and
pick the best sentences.
\citeauthor{das-2022-tfidf}~\cite{das-2022-tfidf} used weighted TF-IDF along with some other feature like sentence position
to rank the sentences.
\citeauthor{chowdhury-etal-2021-tfidf-clustering}~\cite{chowdhury-etal-2021-tfidf-clustering} however used TF-IDF
matrix of a document to build a graph and perform Hierarchical Clustering to group sentences together and pick one
sentence from each group.
However, this model faced the problem of TF-IDF matrix not being semantically equivalent to the actual sentences.
So it didn't perfectly represent the sentences' semantic closeness in the graph.
Using Word Vector Embedding for Bengali solved this problem after The FastText~\cite{grave-etal-2018-fasttext} dataset
came out that had word vector embedding for 157 languages including Bengali.
Using this dataset, \citeauthor{roychowdhury-etal-2022-spectral-base}~\cite{roychowdhury-etal-2022-spectral-base}
proposed a model where they replaced all the words with their respective vector, then averaged the vectors in a
sentence to get the vector for a sentence.
The Gaussian Similarity between the vectors are used to build the graph.
On the graph, spectral clustering was used to group them together and pick one sentence from each cluster using cosine
similarity to get the summary.\\

\begin{figure}
    \centering
    \begin{tikzpicture}[scale=0.001\textwidth]
    \newcommand{\xleft}{2};
    \newcommand{\ytop}{11};
    \newcommand{\xright}{14};
    \newcommand{\ybot}{0};
    % Draw quarters
    \draw (\xleft,\ytop) rectangle (\xleft+10,\ytop+10);
    \draw (\xright,\ytop) rectangle (\xright+10,\ytop+10);
    \draw (\xleft,\ybot) rectangle (\xleft+10,\ybot+10);
    \draw (\xright,\ybot) rectangle (\xright+10,\ybot+10);

    % Add labels
    \node at (0.95,16.15) {(a)};
    \node at (12.95,16.15) {(b)};
    \node at (0.95,5.15) {(c)};
    \node at (12.95,5.15) {(d)};

    %set a
    \newcommand{\aax}{0.25};\newcommand{\aay}{0.35};
    \newcommand{\abx}{0.25};\newcommand{\aby}{5.8};
    \newcommand{\acx}{5};\newcommand{\acy}{0.5};
    \newcommand{\adx}{4.5};\newcommand{\ady}{9.5};
    \newcommand{\aex}{5.5};\newcommand{\aey}{9.6};
    \newcommand{\afx}{9.5};\newcommand{\afy}{0.5};
    %set b
    \newcommand{\bax}{1.5};\newcommand{\bay}{4};
    \newcommand{\bbx}{4.5};\newcommand{\bby}{8.5};
    \newcommand{\bcx}{6.5};\newcommand{\bcy}{1};
    \newcommand{\bdx}{8.2};\newcommand{\bdy}{7.8};
    %set c
    \newcommand{\cax}{2.8};\newcommand{\cay}{4};
    \newcommand{\cbx}{4.2};\newcommand{\cby}{6.2};
    \newcommand{\ccx}{6.2};\newcommand{\ccy}{5};
    \newcommand{\cdx}{6.5};\newcommand{\cdy}{4};

    %set a at a
    \fill[blue] (\xleft+\aax,\ytop+\aay) circle(.15);
    \fill[blue] (\xleft+\abx,\ytop+\aby) circle(.15);
    \fill[blue] (\xleft+\acx,\ytop+\acy) circle(.15);
    \fill[blue] (\xleft+\adx,\ytop+\ady) circle(.15);
    \fill[blue] (\xleft+\aex,\ytop+\aey) circle(.15);
    \fill[blue] (\xleft+\afx,\ytop+\afy) circle(.15);
    %set a at b
    \fill[blue] (\xright+\aax,\ytop+\aay) circle(.15);
    \fill[blue] (\xright+\abx,\ytop+\aby) circle(.15);
    \fill[blue] (\xright+\acx,\ytop+\acy) circle(.15);
    \fill[blue] (\xright+\adx,\ytop+\ady) circle(.15);
    \fill[blue] (\xright+\aex,\ytop+\aey) circle(.15);
    \fill[blue] (\xright+\afx,\ytop+\afy) circle(.15);
    %set a at c
    \fill[blue] (\xleft+\aax,\ybot+\aay) circle(.15);
    \fill[blue] (\xleft+\abx,\ybot+\aby) circle(.15);
    \fill[blue] (\xleft+\acx,\ybot+\acy) circle(.15);
    \fill[blue] (\xleft+\adx,\ybot+\ady) circle(.15);
    \fill[blue] (\xleft+\aex,\ybot+\aey) circle(.15);
    \fill[blue] (\xleft+\afx,\ybot+\afy) circle(.15);

    \newcommand{\amx}{\aax/6+\abx/6+\acx/6+\adx/6+\aex/6+\afx/6};
    \newcommand{\amy}{\aay/6+\aby/6+\acy/6+\ady/6+\aey/6+\afy/6};
    \draw[blue] (\xleft+\amx,\ybot+\amy) circle (0.15);
    %set a at d
    \fill[blue] (\xright+\aax,\ybot+\aay) circle(.15);
    \fill[blue] (\xright+\abx,\ybot+\aby) circle(.15);
    \fill[blue] (\xright+\acx,\ybot+\acy) circle(.15);
    \fill[blue] (\xright+\adx,\ybot+\ady) circle(.15);
    \fill[blue] (\xright+\aex,\ybot+\aey) circle(.15);
    \fill[blue] (\xright+\afx,\ybot+\afy) circle(.15);

    \draw[blue] (\xright+\amx,\ybot+\amy) circle (0.15);
    %set b at a
    \fill[red] (\xleft+\bax,\ytop+\bay) circle(.15);
    \fill[red] (\xleft+\bbx,\ytop+\bby) circle(.15);
    \fill[red] (\xleft+\bcx,\ytop+\bcy) circle(.15);
    \fill[red] (\xleft+\bdx,\ytop+\bdy) circle(.15);
    %set b at c
    \fill[red] (\xleft+\bax,\ybot+\bay) circle(.15);
    \fill[red] (\xleft+\bbx,\ybot+\bby) circle(.15);
    \fill[red] (\xleft+\bcx,\ybot+\bcy) circle(.15);
    \fill[red] (\xleft+\bdx,\ybot+\bdy) circle(.15);

    \newcommand{\bmx}{\bax/4+\bbx/4+\bcx/4+\bdx/4};
    \newcommand{\bmy}{\bay/4+\bby/4+\bcy/4+\bdy/4};
    \draw[red] (\xleft+\bmx,\ybot+\bmy) circle (0.15);
    %set c at b
    \fill[red] (\xright+\cax,\ytop+\cay) circle(.15);
    \fill[red] (\xright+\cbx,\ytop+\cby) circle(.15);
    \fill[red] (\xright+\ccx,\ytop+\ccy) circle(.15);
    \fill[red] (\xright+\cdx,\ytop+\cdy) circle(.15);
    %set c at d
    \fill[red] (\xright+\cax,\ybot+\cay) circle(.15);
    \fill[red] (\xright+\cbx,\ybot+\cby) circle(.15);
    \fill[red] (\xright+\ccx,\ybot+\ccy) circle(.15);
    \fill[red] (\xright+\cdx,\ybot+\cdy) circle(.15);

    \newcommand{\cmx}{\cax/4+\cbx/4+\ccx/4+\cdx/4};
    \newcommand{\cmy}{\cay/4+\cby/4+\ccy/4+\cdy/4};
    \draw[red] (\xright+\cmx,\ybot+\cmy) circle (0.15);

    %drawing line
    \draw[green] (\xleft+\amx,\ybot+\amy) --  (\xleft+\bmx,\ybot+\bmy);
    \draw[green] (\xright+\amx,\ybot+\amy) --  (\xright+\cmx,\ybot+\cmy);

\end{tikzpicture}
    \caption{Scenarios where averaging method fails.}
    \label{fig:sarkar-problem}
\end{figure}

But this model also had a critical weakness.
Words in a sentence are not semantically related instead they are complementary.
So word averages always tend to be in the center and doesn't represent the semantic similarity anymore because the word
vectors get scattered throughout the vector space.
An example is shown in Figure~\ref{fig:sarkar-problem} where the distance between the average word vectors is being
misleading.
In the figure, each point represents a word vector.
The words from the same sentence are grouped together by being colored the same.
In Figure~\ref{fig:sarkar-problem}(a), a scenario is shown in which the words of the two sentences are closer together.
The average distance between these two sentences can be seen in the Figure~\ref{fig:sarkar-problem}(c).
In Figure~\ref{fig:sarkar-problem}(b), a different scenario is shown where the word vectors are farther apart.
The Figure~\ref{fig:sarkar-problem}(d) shows the average vector for these two sentences.
Although the words in the latter scenario are farther apart, their average is shown to be closer than in the first
scenario, thus making this metric misleading.