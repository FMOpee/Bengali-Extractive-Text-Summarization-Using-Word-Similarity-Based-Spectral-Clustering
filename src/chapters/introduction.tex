Text Summarization is the process of shortening a larger text without losing any key information.
Automatic Text Summarization (ATS) is the process of summarizing a document
automatically~\cite{Widyassari-2022-rev-ats-tech-met}.
This is a major area of natural language processing (NLP) research which is progressing very rapidly.
This task has become more and more important in the current digital age, where the amount of textual
data grows exponentially in many fields~\cite{2015-Forbes-80-created-last-2-years},
such as news, legal documents, health reports, research papers, and social media content.
ATS techniques allow users to quickly get the essential information without needing to read through large
amounts of text~\cite{wafaa-2021-summary-comprehensive-review}.
Automatic Text summarization is used in many fields, from automatic news summarization, content filtering, and recommendation
systems to assisting legal professionals in going through long documents and researchers in reviewing academic papers.
It can also play a critical role in personal assistants and chatbots, providing condensed information to users quickly
and efficiently~\cite{tas-2017-rev-text-sum-2}.\\

There are two main types of automatic text summarization: extractive and abstractive~\cite{tas-2017-rev-text-sum-2}.
Extractive summarization, which is the focus of this paper, works by selecting best sentences or phrases directly
from the source document, maintaining the original wording and sentence structure~\cite{moratanch-2017-extractive-review}.
In contrast, abstractive summarization involves generating new sentences to capture the meaning of the
text, Similar to human made summarization~\cite{Moratanch-2016-abstractive-rev}.
Extractive summarization is widely used because of its simplicity and effectiveness, especially for languages with
limited NLP resources~\cite{gupta-2010-extractive-rev}.\\

There are a lot of different ways to achieve extractive summarization.
A commonly used method for extractive text summarization is graph-based summarization.
This method represents the sentences of a document as nodes of a graph, and the edges between them are weighted by the
similarity between the sentences~\cite{wafaa-2021-summary-comprehensive-review}.
Popular algorithms like LexRank~\cite{Erkan-lexRank-2004} and TextRank~\cite{mihalcea-2004-textrank} build graphs based
on cosine similarity between sentence embeddings and apply ranking algorithms such as PageRank~\cite{page-PageRank-1999} in
case of LexRank~\cite{Erkan-lexRank-2004} or Random Walk in case of TextRank~\cite{mihalcea-2004-textrank} to
determine which sentences are the most important.
These sentences are then selected to make the summary.
Graph-based methods offer a more robust way to capture sentence importance and relationship, ensuring that the
extracted summary covers the key information while minimizing redundancy~\cite{wafaa-2021-summary-comprehensive-review}.\\

A subset of graph-based approach to extractive summarization is clustering-based summarization.
Here, sentences are grouped into clusters based on their semantic similarity, and one representative sentence from each
cluster is chosen to form the summary~\cite{Mohan-2022-topic-modeling-rev-clustering}.
Clustering reduces redundancy by ensuring that similar sentences are grouped together and that only the most
representative sentence is selected.
This method is effective in documents that cover multiple topics or subtopics, as it allows the summary to touch on
each area without being repetitive.\\

For Bengali, a low-resource language, early attempts at text summarization relied on traditional methods like TF-IDF
(Term Frequency-Inverse Document Frequency) scoring~\cite{das-2022-tfidf,sarkar-2012-tfidf}.
These approaches, while simple, faced challenges in capturing the true meaning of sentences, as they treated words as
isolated terms~\cite{tas-2017-rev-text-sum-2}.
Graph-based methods introduced improvements by incorporating sentence similarity, but they were still limited by the
quality of the embeddings used for the Bengali language.
With the advent of word embedding models like FastText~\cite{grave-etal-2018-fasttext}, which supports over 157
languages, including Bengali, it became possible to represent words in a Vector Space Model, thus enabling
more accurate sentence similarity calculations.\\

However, existing models that use word embeddings, such as \citeauthor{roychowdhury-etal-2022-spectral-base}’s
\cite{roychowdhury-etal-2022-spectral-base} Sentence Average Similarity-based Spectral Clustering (SASbSC) method,
encountered issues when averaging word vectors to represent sentence meaning.
This method failed in most cases because words in a sentence are often complementary rather than being similar, leading
to inaccurate sentence representations when averaging their vectors.
As a result, important word-to-word relationships between sentences were lost, reducing the effectiveness of the method.\\

In this paper, we propose a new approach to address these challenges.
Our method improves upon previous attempts~\cite{roychowdhury-etal-2022-spectral-base}
by focusing on the individual similarity between words in sentences rather than averaging word vectors.
Here the gaussian similarities between each word and the Most Similar Word from the other sentence for that word are
used to get the similarity between the two sentence.
This method captures the similarity of meaning between two sentences more accurately.
By applying Gaussian similarity to the Most Similar Word Distance ($D_{msw}$) values, we build an affinity matrix that
better reflects sentence closeness which can be proved by the effectiveness of the model (Table~\ref{tab:result_comparison-1}).
We then apply spectral clustering on this matrix to group similar sentences together and use TF-IDF to select the
most representative sentences from each cluster.
This approach reduces redundancy and improves the quality of the summary by selecting sentences that are not only
relevant but also diverse.
This method works really well for Bengali on four diverse datasets consistently (Figure~\ref{fig:radarchart}).
It consistently outperforms other graph based methods like BenSumm~\cite{chowdhury-etal-2021-tfidf-clustering},
SASbSC~\cite{roychowdhury-etal-2022-spectral-base}, LexRank~\cite{Erkan-lexRank-2004}.
It also performs similarly well on other low resource languages we tried it on.
These languages are Hindi, Marathi and Turkish (Table~\ref{tab:other_language}).
These are the only other low resource languages where we found reliable evaluation datasets and tested our model on them.
The search process was not exhaustive due to our language barrier.\\

The main contributions of this paper are:
(I) Creating a new way to calculate similarity between two sentences.
(II) Contributes a novel methodology for extractive text summarization for the Bengali language.
By improving sentence similarity calculations and enhancing clustering techniques.
(III) It addresses the limitations of previous models, such as misleading word vector averages~\cite{roychowdhury-etal-2022-spectral-base},
(IV) It offers a generalizable solution for creating less redundant and information rich summaries across languages.
(V) It provides a publicly available high quality dataset of 500 human generated summaries.\\

The rest of the paper is organized as follows:
The Literature review and Methodology are described in section 2 and 3 respectively.
The section 4 illustrates the findings of this work.
The section 5 discusses the findings of the paper in more depth.